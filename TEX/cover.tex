
%%% Local Variables:
%%% mode: latex
%%% TeX-master: t
%%% End:
% \secretlevel{绝密} \secretyear{2100}

\ctitle{基于化学反应的多功能光子晶体体系的研究}
% 根据自己的情况选,不用这样复杂
\makeatletter
\ifthu@bachelor\relax\else
  \ifthu@doctor
    \cdegree{理学博士}
  \else
    \ifthu@master
      \cdegree{理学硕士}
    \fi
  \fi
\fi
\makeatother


\cdepartment[化学]{化学系}
\cmajor{化学}
\cauthor{田天} 
\csupervisor{李广涛教授}
% 如果没有副指导老师或者联合指导老师,把下面两行相应的删除即可。
% \cassosupervisor{陈文光教授}
% \ccosupervisor{某某某教授}
% 日期自动生成,如果你要自己写就改这个cdate
\cdate{二〇一五年五月}

% 博士后部分
% \cfirstdiscipline{计算机科学与技术}
% \cseconddiscipline{系统结构}
% \postdoctordate{2009年7月——2011年7月}

\etitle{Research of Multifunctional Photonic Crystal Systems Based on Chemical Reaction} 
% 这块比较复杂,需要分情况讨论:
% 1. 学术型硕士
%    \edegree:必须为Master of Arts或Master of Science(注意大小写)
%              “哲学、文学、历史学、法学、教育学、艺术学门类,公共管理学科
%               填写Master of Arts,其它填写Master of Science”
%    \emajor:“获得一级学科授权的学科填写一级学科名称,其它填写二级学科名称”
% 2. 专业型硕士
%    \edegree:“填写专业学位英文名称全称”
%    \emajor:“工程硕士填写工程领域,其它专业学位不填写此项”
% 3. 学术型博士
%    \edegree:Doctor of Philosophy(注意大小写)
%    \emajor:“获得一级学科授权的学科填写一级学科名称,其它填写二级学科名称”
% 4. 专业型博士
%    \edegree:“填写专业学位英文名称全称”
%    \emajor:不填写此项
\edegree{Master of Science} 
\emajor{Chemistry} 
\eauthor{Tian Tian} 
\esupervisor{Professor Guangtao Li} 
% \eassosupervisor{Chen Wenguang} 
% 这个日期也会自动生成,你要改么?
\edate{May, 2005}

% 定义中英文摘要和关键字
%%% TODO
%%%
\begin{cabstract}
  光子晶体是一类具有周期性介电常数结构的材料。独特的周期性结构使得光子晶体能够对光的传播进行调制。
  独特的内部结构赋予了光子晶体信号自表达的特性。这种性质使光子晶体在光学调控、化学传感、生物工程等方面具有广泛的应用。

  光子晶体独特的结构与光学特性使其成为一个优秀的平台,可以与多种功能材料相结合。将光子晶体与化学反应相结合是近年来光子晶体研究的前沿热点。化学反应的多样性不仅拓宽光子晶体的传感应用领域,更赋予了光子晶体极大的拓展性。化学反应与光子晶体的光学特性、孔道特性以及信号自表达特性的有机结合,能够开发新颖的多功能材料。

  本文探究了几种光子晶体结合与不同的化学反应结合所形成的新型多功能材料。首先,基于马来酰亚胺-巯基的特异性反应与光子晶体的信号自表达特性,发展了一种高灵敏的乙酰胆碱酯酶活性传感平台。该平台具有较低的乙酰胆碱酯酶检出极限;同时,受惠于光子晶体便利的操作与裸眼观察的特性,这种光子晶体平台能够适用于多场合的实际应用。此外,基于光敏高分子的选择性脱保护及化学修饰,发展了一种平面型反蛋白石光子晶体的图案化方法。由于化学反应的选择性,这种光子晶体的图案化修饰方法无须复杂的物理操作,能够保持反蛋白石结构的完整性。更为重要的是此方法形成了二维尺度上的化学复杂度,结合光子晶体性质能够实现诸如二维亲疏水梯度结构、动态可调节的图案等复杂应用。最后,我们将二维尺度上的化学复杂度拓展到三维空间中,利用光子晶体微球的扩散刻蚀特性,发展了一种基于刻蚀-反应的光子晶体微球多功能结构的制备方法。该方法具有很高的结构可调性与多样性。而三维尺度上的各向异性化学组成与光子晶体性质结合能产生复杂的功能体系,包括三维尺度的亲疏水梯度结构、正交化学反应平台以及具有双重信号表达的级联反应体系等,大大丰富了光子晶体材料的应用领域。

  本文围绕化学反应与光子晶体的协同效应,从各向同性材料发展到二维乃至三维尺度上的各向异性光子晶体材料,证实了化学反应与光子晶体的有机结合能够实现非常丰富的功能。基于化学反应与光子晶体两者兼有的多样性与拓展性,可在本文的研究基础上发展更多复杂功能体系,具有非常广阔的研究及应用前景。

\end{cabstract}

\ckeywords{光子晶体;化学反应;多功能材料;各向异性}

\begin{eabstract} 
   Photonic crystal is a class of materials with periodic dielectric constant. The intrinsic periodicity allows tuning of the passage of light, which contributes to its unique property of photonic bandgap. As the unique optical properties of photonic crystal rise from its inner structure, it is endowed with the mighty ability of signal self-reporting, which ensures its vast applications in optical control, chemical sensing and bioengineering.

   The photonic structure together with its optical properties promote photonic crystal to be a universal platform, which could combine a large variety of functional materials. The combination of photonic crystal with chemical reaction is the recent hotspot of research. The variety of chemical reaction enriches the application of photonic crystal and its extendibility. With the coherent combination of chemical reaction and photonic crystal, it is promising to develop novel multi-functional materials.

   The research in this thesis mainly focuses on several chemical reactions and their contribution to develop photonic-based novel materials. The maleimide-thio reaction is first studied to develop a highly sensitive sensing platform for the activity of acetylcholinesterase. The approach owns remarkably low limit of detection, and is also promising for its facile manipulation and bare-eye observation, which is highly potent in real applications. Besides, a patterning approach for inverse opal material is developed, based on the selective deprotection-modification approach of photolabile polymer. Such method owns the advantage of facile manipulation and high selectivity, compared with cumbersome physical modification. More important, 2D pattern with chemical complexity is developed. The deprotection-modification approach can be further applied to fabricate complex systems, including hydrophilicity gradient or dynamic chemical pattern for anti-counterfeiting materials. More interestingly, taking advantage of the diffusive etching of photonic microbeads, an alternative approach of fabricating 3D chemical complexity was approached, based on the etch-reaction process. 
   The spatial anisotropy and optical properties of photonic crystal can emerge novel functional systems, including 3D hydrophilicity gradient, chemoorthogonal reaction platform as well as cascade enzyme reaction system with dual output. 

   This thesis follows the routine of studying the combination of chemical reaction and photonic crystal, from isotropic material, to 2D and eventually 3D anisotropic complex photonic crystal material. The research proved that the combination of chemical reaction with photonic crystal is highly promising for both variety and extendibility. It is hopeful that more complex functional systems can be developed on the basis of this study.
\end{eabstract}

\ekeywords{photonic crystal, chemical reaction, multi-functional material, anisotropy}
