%%% Local Variables:
%%% mode: latex
%%% TeX-master: t
%%% End:

\chapter{总结与展望}
\label{ch:conclusion}

\section{本文总结}

本文主要研究了化学反应与光子晶体相结合产生的新型多功能材料及其应用。光子晶体自身的结构-光信号转换特性使其成为一种非常理想的信号自表达平台。而化学反应自身的多样性又对光子晶体的拓展性有所裨益。基于化学反应,可以实现对一系列目标化合物的特异性识别,从而发展光子晶体的化学传感应用;而同时,化学反应也使得光子晶体材料成为具有广泛后修饰可能性的平台。通过精心设计的化学修饰方法,结合光子晶体自身的禁带特性及内部孔道特性,能够实现相当具有创造性的应用。

首先,我们通过马来酰亚胺-巯基反应发展了信号自表达的乙酰胆碱酯酶活性传感光子晶体材料。这种光子晶体检测方法具有很高的灵敏性,以及方法灵活,操作简便,信号自表达等诸多优点。此外,这种基于化学反应的检测平台还能够适用于活性检测、酶动力学测量、抑制剂筛选等多场合应用,证明了化学反应与光子晶体结合带来的优异性能及功能材料拓展性。

同时,我们尝试利用化学反应的选择性来突破光子晶体材料的各向同性,来打造具有各向异性化学组成的光子晶体材料。首先,我们基于光敏高分子实现了平面反蛋白石光子晶体的图案化与复杂化学组成修饰等应用。利用NVOC保护基在紫外光照下可控脱保护与修饰的特性,发展了一种在平面型反蛋白石光子晶体材料上图案化的方法。在图案化的同时,带来了二维尺度上的化学复杂性结构。基于这种特征结构,我们发展了诸如二维尺度上的亲疏水梯度、动态调控的图案等应用,为光子晶体的实用化提供了一些思路。

最后,我们将这种层次化修饰方法拓展到三维空间中。基于选择性刻蚀-反应实现了在光子晶体微球上的三维化学修饰与功能化。基于光子晶体光学性质、孔洞结构及三维复杂化学组成的协同作用发展了一种可高度拓展的光子晶体平台。
在这种光子晶体微球平台上,基于选择性刻蚀-反应实现了三维尺度上的亲疏水梯度、正交化学反应微球、自表达级联反应体系等应用。
展示了结构、化学组成各向异性的光子晶体微球在复杂功能体系中的潜在应用。

总结全文,我们研究了化学反应在不同层面上与光子晶体相结合所带来的优异性质。从各向同性的灵敏传感单元到平面图案化的防伪材料再发展为具有立体化学复杂性的多功能协同体系,可以看出化学反应极大地拓展了光子晶体的应用领域。相信基于本文的工作能够有一系列更为优秀的光子晶体材料被开发出来。

\section{展望}

不难注意到,本文中所利用的化学反应基本都是基于高分子材料展开的。尽管能够实现相当多样的功能,但一些特殊功能,例如磁性、催化活性、机械强度方法仍然无法与无机材料或复合材料媲美。未来的基于化学反应的光子晶体将更多地研究无机光子晶体材料的可控修饰方法,以及向有机高分子体系中引入具有活性的无机组分。这样的无机-有机整合,加上光子晶体独特的光学及结构特性,有望发展更为优异的功能材料。